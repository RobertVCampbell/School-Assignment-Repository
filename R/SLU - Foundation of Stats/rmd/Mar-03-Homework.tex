% Options for packages loaded elsewhere
\PassOptionsToPackage{unicode}{hyperref}
\PassOptionsToPackage{hyphens}{url}
%
\documentclass[
]{article}
\usepackage{lmodern}
\usepackage{amsmath}
\usepackage{ifxetex,ifluatex}
\ifnum 0\ifxetex 1\fi\ifluatex 1\fi=0 % if pdftex
  \usepackage[T1]{fontenc}
  \usepackage[utf8]{inputenc}
  \usepackage{textcomp} % provide euro and other symbols
  \usepackage{amssymb}
\else % if luatex or xetex
  \usepackage{unicode-math}
  \defaultfontfeatures{Scale=MatchLowercase}
  \defaultfontfeatures[\rmfamily]{Ligatures=TeX,Scale=1}
\fi
% Use upquote if available, for straight quotes in verbatim environments
\IfFileExists{upquote.sty}{\usepackage{upquote}}{}
\IfFileExists{microtype.sty}{% use microtype if available
  \usepackage[]{microtype}
  \UseMicrotypeSet[protrusion]{basicmath} % disable protrusion for tt fonts
}{}
\makeatletter
\@ifundefined{KOMAClassName}{% if non-KOMA class
  \IfFileExists{parskip.sty}{%
    \usepackage{parskip}
  }{% else
    \setlength{\parindent}{0pt}
    \setlength{\parskip}{6pt plus 2pt minus 1pt}}
}{% if KOMA class
  \KOMAoptions{parskip=half}}
\makeatother
\usepackage{xcolor}
\IfFileExists{xurl.sty}{\usepackage{xurl}}{} % add URL line breaks if available
\IfFileExists{bookmark.sty}{\usepackage{bookmark}}{\usepackage{hyperref}}
\hypersetup{
  pdftitle={Homework \#04},
  pdfauthor={Robert Campbell},
  hidelinks,
  pdfcreator={LaTeX via pandoc}}
\urlstyle{same} % disable monospaced font for URLs
\usepackage[margin=1in]{geometry}
\usepackage{color}
\usepackage{fancyvrb}
\newcommand{\VerbBar}{|}
\newcommand{\VERB}{\Verb[commandchars=\\\{\}]}
\DefineVerbatimEnvironment{Highlighting}{Verbatim}{commandchars=\\\{\}}
% Add ',fontsize=\small' for more characters per line
\usepackage{framed}
\definecolor{shadecolor}{RGB}{248,248,248}
\newenvironment{Shaded}{\begin{snugshade}}{\end{snugshade}}
\newcommand{\AlertTok}[1]{\textcolor[rgb]{0.94,0.16,0.16}{#1}}
\newcommand{\AnnotationTok}[1]{\textcolor[rgb]{0.56,0.35,0.01}{\textbf{\textit{#1}}}}
\newcommand{\AttributeTok}[1]{\textcolor[rgb]{0.77,0.63,0.00}{#1}}
\newcommand{\BaseNTok}[1]{\textcolor[rgb]{0.00,0.00,0.81}{#1}}
\newcommand{\BuiltInTok}[1]{#1}
\newcommand{\CharTok}[1]{\textcolor[rgb]{0.31,0.60,0.02}{#1}}
\newcommand{\CommentTok}[1]{\textcolor[rgb]{0.56,0.35,0.01}{\textit{#1}}}
\newcommand{\CommentVarTok}[1]{\textcolor[rgb]{0.56,0.35,0.01}{\textbf{\textit{#1}}}}
\newcommand{\ConstantTok}[1]{\textcolor[rgb]{0.00,0.00,0.00}{#1}}
\newcommand{\ControlFlowTok}[1]{\textcolor[rgb]{0.13,0.29,0.53}{\textbf{#1}}}
\newcommand{\DataTypeTok}[1]{\textcolor[rgb]{0.13,0.29,0.53}{#1}}
\newcommand{\DecValTok}[1]{\textcolor[rgb]{0.00,0.00,0.81}{#1}}
\newcommand{\DocumentationTok}[1]{\textcolor[rgb]{0.56,0.35,0.01}{\textbf{\textit{#1}}}}
\newcommand{\ErrorTok}[1]{\textcolor[rgb]{0.64,0.00,0.00}{\textbf{#1}}}
\newcommand{\ExtensionTok}[1]{#1}
\newcommand{\FloatTok}[1]{\textcolor[rgb]{0.00,0.00,0.81}{#1}}
\newcommand{\FunctionTok}[1]{\textcolor[rgb]{0.00,0.00,0.00}{#1}}
\newcommand{\ImportTok}[1]{#1}
\newcommand{\InformationTok}[1]{\textcolor[rgb]{0.56,0.35,0.01}{\textbf{\textit{#1}}}}
\newcommand{\KeywordTok}[1]{\textcolor[rgb]{0.13,0.29,0.53}{\textbf{#1}}}
\newcommand{\NormalTok}[1]{#1}
\newcommand{\OperatorTok}[1]{\textcolor[rgb]{0.81,0.36,0.00}{\textbf{#1}}}
\newcommand{\OtherTok}[1]{\textcolor[rgb]{0.56,0.35,0.01}{#1}}
\newcommand{\PreprocessorTok}[1]{\textcolor[rgb]{0.56,0.35,0.01}{\textit{#1}}}
\newcommand{\RegionMarkerTok}[1]{#1}
\newcommand{\SpecialCharTok}[1]{\textcolor[rgb]{0.00,0.00,0.00}{#1}}
\newcommand{\SpecialStringTok}[1]{\textcolor[rgb]{0.31,0.60,0.02}{#1}}
\newcommand{\StringTok}[1]{\textcolor[rgb]{0.31,0.60,0.02}{#1}}
\newcommand{\VariableTok}[1]{\textcolor[rgb]{0.00,0.00,0.00}{#1}}
\newcommand{\VerbatimStringTok}[1]{\textcolor[rgb]{0.31,0.60,0.02}{#1}}
\newcommand{\WarningTok}[1]{\textcolor[rgb]{0.56,0.35,0.01}{\textbf{\textit{#1}}}}
\usepackage{graphicx}
\makeatletter
\def\maxwidth{\ifdim\Gin@nat@width>\linewidth\linewidth\else\Gin@nat@width\fi}
\def\maxheight{\ifdim\Gin@nat@height>\textheight\textheight\else\Gin@nat@height\fi}
\makeatother
% Scale images if necessary, so that they will not overflow the page
% margins by default, and it is still possible to overwrite the defaults
% using explicit options in \includegraphics[width, height, ...]{}
\setkeys{Gin}{width=\maxwidth,height=\maxheight,keepaspectratio}
% Set default figure placement to htbp
\makeatletter
\def\fps@figure{htbp}
\makeatother
\setlength{\emergencystretch}{3em} % prevent overfull lines
\providecommand{\tightlist}{%
  \setlength{\itemsep}{0pt}\setlength{\parskip}{0pt}}
\setcounter{secnumdepth}{-\maxdimen} % remove section numbering
\ifluatex
  \usepackage{selnolig}  % disable illegal ligatures
\fi

\title{Homework \#04}
\author{Robert Campbell}
\date{03 Mar 2021}

\begin{document}
\maketitle

\hypertarget{chapter-03}{%
\subsection{Chapter 03}\label{chapter-03}}

\begin{Shaded}
\begin{Highlighting}[]
\NormalTok{trials }\OtherTok{\textless{}{-}} \DecValTok{100000}
\end{Highlighting}
\end{Shaded}

\hypertarget{problem-18}{%
\subsubsection{Problem 18}\label{problem-18}}

\begin{Shaded}
\begin{Highlighting}[]
\NormalTok{x }\OtherTok{\textless{}{-}} \FunctionTok{rgeom}\NormalTok{(trials,.}\DecValTok{09}\NormalTok{)}
\NormalTok{y }\OtherTok{\textless{}{-}}\NormalTok{ x }\SpecialCharTok{+} \DecValTok{1}
\FunctionTok{mean}\NormalTok{(y}\SpecialCharTok{==}\DecValTok{20}\NormalTok{)}
\end{Highlighting}
\end{Shaded}

\begin{verbatim}
## [1] 0.01447
\end{verbatim}

\hypertarget{problem-30}{%
\subsubsection{Problem 30}\label{problem-30}}

\(\frac{Var(eN)}{E[eN]} = \frac{e^2*Var(N)}{e*E[N]} = \frac{eN}{N} = e\)

\hypertarget{problem-31}{%
\subsubsection{Problem 31}\label{problem-31}}

****Requires infinite of e\^{}x 1 + x + x\^{}2 / 2! + x\^{}3 / 3! +
\ldots{}

\hypertarget{chapter-04}{%
\subsection{Chapter 04}\label{chapter-04}}

\hypertarget{problem-01}{%
\subsubsection{Problem 01}\label{problem-01}}

\begin{enumerate}
\def\labelenumi{\alph{enumi}.}
\tightlist
\item
  \(P(X \geq0.5)=\int_.5^1 2x\ dx = \frac{1^2 * 2}{2} - \frac{0.5^2 * 2}{2} = 1 - .25 = .75\)
\item
  \(P(X\geq 0.5 | X\geq0.25) = \frac{P(X\geq 0.5 \& X\geq0.25)}{P(X\geq0.25)}\)
  \(\frac{\int_.5^1 2x\ dx}{\int_.25^1 2x\ dx} = \frac{0.75}{1 - 0.25^2} = \frac{.75}{0.9375} = 0.8\)
\end{enumerate}

\hypertarget{problem-02}{%
\subsubsection{Problem 02}\label{problem-02}}

\(1 = \int_0^1 Cx^2\ dx + \int_1^2C(2-x)^2\ dx = C*[\int_0^1x^2\ dx + \int_1^2(2-x)^2\ dx]\)
\(1 = C*[\frac{1}{3} + \frac{1}{3}]\) \(C = \frac{3}{2}\)

\hypertarget{problem-06}{%
\subsubsection{Problem 06}\label{problem-06}}

\begin{enumerate}
\def\labelenumi{\alph{enumi}.}
\tightlist
\item
  \(1 = \int_0^13*(1-x)^2\ dx = 3 * -\int_1^0u^2\ du = 3 * [\frac{1}{3} - \frac{0}{3}] = 1\)
\item
  \(Mean = \int_0^1 3x(1-x)^2\ dx = 0.25\) \$Var = \int\_0\^{}1
  3x\textsuperscript{2(1-x)}2~dx - Mean\^{}2 = 0.1 - 0.0625 = 0.0375 \$
\item
  \(P(x \leq 0.5) = \int_0^.5 3*(1-x)^2\ dx = 0.875\)
\item
  \(P(x\leq 0.5 | x \geq 0.25)= \frac{P(x\leq 0.5\ \&\ x \geq 0.25)}{P(x\geq0.25)} = \frac{\int_.25^.5 3*(1-x)^2\ dx}{\int_.25^1 3*(1-x)^2\ dx}\)
  \(\frac{\frac{19}{64}}{\frac{27}{64}} = \frac{19}{27} = 0.703704\)
\end{enumerate}

\hypertarget{problem-07}{%
\subsubsection{Problem 07}\label{problem-07}}

\(Var(x) = 3\) \(Var(2x+1) = 2^2 * Var(X) + Var(1) = 4 * 3 + 0 = 12\)

\hypertarget{problem-08}{%
\subsubsection{Problem 08}\label{problem-08}}

\begin{enumerate}
\def\labelenumi{\alph{enumi}.}
\tightlist
\item
  Estimate mean: 1
\item
  Estimate sd: 1
\item
  a = 0
\item
  Lower bound estimate \$P(0 \textless= x \textless=2): 0.225 * 2 = 0.45
  \$ \#\#\# Problem 11
\item
\end{enumerate}

\begin{Shaded}
\begin{Highlighting}[]
\DecValTok{1} \SpecialCharTok{{-}} \FunctionTok{pnorm}\NormalTok{(}\DecValTok{85}\NormalTok{, }\DecValTok{80}\NormalTok{, }\DecValTok{5}\NormalTok{)}
\end{Highlighting}
\end{Shaded}

\begin{verbatim}
## [1] 0.1586553
\end{verbatim}

\begin{enumerate}
\def\labelenumi{\alph{enumi}.}
\setcounter{enumi}{1}
\tightlist
\item
\end{enumerate}

\begin{Shaded}
\begin{Highlighting}[]
  \FunctionTok{mean}\NormalTok{(}\FunctionTok{replicate}\NormalTok{(trials, }\FunctionTok{sum}\NormalTok{(}\FunctionTok{rnorm}\NormalTok{(}\DecValTok{10}\NormalTok{,}\DecValTok{80}\NormalTok{,}\DecValTok{5}\NormalTok{)}\SpecialCharTok{\textgreater{}=}\DecValTok{85}\NormalTok{))}\SpecialCharTok{\textgreater{}}\DecValTok{3}\NormalTok{)}
\end{Highlighting}
\end{Shaded}

\begin{verbatim}
## [1] 0.05974
\end{verbatim}

\hypertarget{problem-12}{%
\subsubsection{Problem 12}\label{problem-12}}

\begin{enumerate}
\def\labelenumi{\alph{enumi}.}
\tightlist
\item
\end{enumerate}

\begin{Shaded}
\begin{Highlighting}[]
\FunctionTok{hist}\NormalTok{(}\FunctionTok{rnorm}\NormalTok{(trials, }\DecValTok{5036}\NormalTok{, }\DecValTok{122}\NormalTok{))}
\end{Highlighting}
\end{Shaded}

\includegraphics{Mar-03-Homework_files/figure-latex/unnamed-chunk-5-1.pdf}
b. 0, the probability of a specific number occuring in a continuous
probability function is 0. The question could be modified to ask "what
is the probability of the rope breaking at more than 5000lbs, but less
than 5001lbs, which could be calculated as:

\begin{Shaded}
\begin{Highlighting}[]
\FunctionTok{pnorm}\NormalTok{(}\DecValTok{5001}\NormalTok{, }\DecValTok{5036}\NormalTok{,}\DecValTok{122}\NormalTok{) }\SpecialCharTok{{-}} \FunctionTok{pnorm}\NormalTok{(}\DecValTok{5000}\NormalTok{, }\DecValTok{5036}\NormalTok{, }\DecValTok{122}\NormalTok{)}
\end{Highlighting}
\end{Shaded}

\begin{verbatim}
## [1] 0.003134462
\end{verbatim}

\begin{enumerate}
\def\labelenumi{\alph{enumi}.}
\setcounter{enumi}{2}
\tightlist
\item
\end{enumerate}

\begin{Shaded}
\begin{Highlighting}[]
\FunctionTok{qnorm}\NormalTok{(.}\DecValTok{95}\NormalTok{,}\DecValTok{5036}\NormalTok{,}\DecValTok{122}\NormalTok{)}
\end{Highlighting}
\end{Shaded}

\begin{verbatim}
## [1] 5236.672
\end{verbatim}

\hypertarget{problem-15}{%
\subsubsection{Problem 15}\label{problem-15}}

\begin{Shaded}
\begin{Highlighting}[]
\NormalTok{x }\OtherTok{\textless{}{-}} \FunctionTok{seq}\NormalTok{(}\DecValTok{0}\NormalTok{,}\DecValTok{1}\NormalTok{,.}\DecValTok{1}\NormalTok{)}
\FunctionTok{plot}\NormalTok{(x, }\FunctionTok{dunif}\NormalTok{(x,}\DecValTok{0}\NormalTok{,}\DecValTok{1}\NormalTok{),}\AttributeTok{type=}\StringTok{\textquotesingle{}l\textquotesingle{}}\NormalTok{) }
\end{Highlighting}
\end{Shaded}

\includegraphics{Mar-03-Homework_files/figure-latex/unnamed-chunk-8-1.pdf}

\begin{Shaded}
\begin{Highlighting}[]
\FunctionTok{plot}\NormalTok{(x, }\FunctionTok{punif}\NormalTok{(x,}\DecValTok{0}\NormalTok{,}\DecValTok{1}\NormalTok{),}\AttributeTok{type=}\StringTok{\textquotesingle{}l\textquotesingle{}}\NormalTok{) }
\end{Highlighting}
\end{Shaded}

\includegraphics{Mar-03-Homework_files/figure-latex/unnamed-chunk-8-2.pdf}

\hypertarget{problem-17}{%
\subsubsection{Problem 17}\label{problem-17}}

\begin{Shaded}
\begin{Highlighting}[]
\NormalTok{x }\OtherTok{\textless{}{-}} \FunctionTok{rexp}\NormalTok{(trials,}\FloatTok{0.25}\NormalTok{)}
\end{Highlighting}
\end{Shaded}

\begin{enumerate}
\def\labelenumi{\alph{enumi}.}
\tightlist
\item
\end{enumerate}

\begin{Shaded}
\begin{Highlighting}[]
\FunctionTok{mean}\NormalTok{(x)}
\end{Highlighting}
\end{Shaded}

\begin{verbatim}
## [1] 4.013208
\end{verbatim}

\begin{enumerate}
\def\labelenumi{\alph{enumi}.}
\setcounter{enumi}{1}
\tightlist
\item
  By inspection, it would be a = 0 and does not contain the mean.
\end{enumerate}

\hypertarget{problem-19}{%
\subsubsection{Problem 19}\label{problem-19}}

\begin{Shaded}
\begin{Highlighting}[]
\NormalTok{c1}\OtherTok{\textless{}{-}} \FunctionTok{rexp}\NormalTok{(trials, }\DecValTok{1}\SpecialCharTok{/}\DecValTok{5}\NormalTok{)}
\NormalTok{c2}\OtherTok{\textless{}{-}} \FunctionTok{rexp}\NormalTok{(trials)}
\FunctionTok{mean}\NormalTok{(}\FunctionTok{pmax}\NormalTok{(c1,c2)}\SpecialCharTok{\textless{}}\DecValTok{10}\NormalTok{)}
\end{Highlighting}
\end{Shaded}

\begin{verbatim}
## [1] 0.86505
\end{verbatim}

\hypertarget{problem-20}{%
\subsubsection{Problem 20}\label{problem-20}}

See attached handwritten work.

\hypertarget{problem-21}{%
\subsubsection{Problem 21}\label{problem-21}}

\begin{enumerate}
\def\labelenumi{\alph{enumi}.}
\tightlist
\item
  It is called the memoryless property because previous events do not
  affect the probability of the next event.
\item
  \(P(X\geq a)\int_a^\infty \lambda e^{-\lambda x}\ dx = \frac{-\lambda e^{-\lambda x}}{\lambda}|_a^\infty\)
  \(0 - (-e^{-a\lambda}) = e^{-a\lambda}\)
\item
\end{enumerate}

\hypertarget{problem-22}{%
\subsubsection{Problem 22}\label{problem-22}}

\begin{enumerate}
\def\labelenumi{\alph{enumi}.}
\tightlist
\item
  Binomial P(Y==3):
\end{enumerate}

\begin{Shaded}
\begin{Highlighting}[]
\FunctionTok{dbinom}\NormalTok{(}\DecValTok{3}\NormalTok{,}\DecValTok{10}\NormalTok{,}\DecValTok{1}\SpecialCharTok{/}\DecValTok{6}\NormalTok{)}
\end{Highlighting}
\end{Shaded}

\begin{verbatim}
## [1] 0.1550454
\end{verbatim}

\begin{Shaded}
\begin{Highlighting}[]
\NormalTok{dice}\OtherTok{\textless{}{-}}\FunctionTok{rbinom}\NormalTok{(trials,}\DecValTok{10}\NormalTok{,}\DecValTok{1}\SpecialCharTok{/}\DecValTok{6}\NormalTok{)}
\FunctionTok{mean}\NormalTok{(dice)}
\end{Highlighting}
\end{Shaded}

\begin{verbatim}
## [1] 1.65879
\end{verbatim}

\begin{Shaded}
\begin{Highlighting}[]
\FunctionTok{var}\NormalTok{(dice)}
\end{Highlighting}
\end{Shaded}

\begin{verbatim}
## [1] 1.39152
\end{verbatim}

\begin{enumerate}
\def\labelenumi{\alph{enumi}.}
\setcounter{enumi}{1}
\tightlist
\item
  Poisson, in accidents per week
\end{enumerate}

\begin{Shaded}
\begin{Highlighting}[]
\FunctionTok{dpois}\NormalTok{(}\DecValTok{2}\NormalTok{,}\DecValTok{2}\NormalTok{)}
\end{Highlighting}
\end{Shaded}

\begin{verbatim}
## [1] 0.2706706
\end{verbatim}

\begin{Shaded}
\begin{Highlighting}[]
\NormalTok{traf}\OtherTok{\textless{}{-}}\FunctionTok{rpois}\NormalTok{(trials,}\DecValTok{2}\NormalTok{)}
\FunctionTok{mean}\NormalTok{(traf)}
\end{Highlighting}
\end{Shaded}

\begin{verbatim}
## [1] 2.00001
\end{verbatim}

\begin{Shaded}
\begin{Highlighting}[]
\FunctionTok{var}\NormalTok{(traf)}
\end{Highlighting}
\end{Shaded}

\begin{verbatim}
## [1] 2.00677
\end{verbatim}

\begin{enumerate}
\def\labelenumi{\alph{enumi}.}
\setcounter{enumi}{2}
\tightlist
\item
  Uniform, \(Mean = \frac{60-0}{2} = 30\)
\item
  Exponential with rate in customers per hour
\end{enumerate}

\begin{Shaded}
\begin{Highlighting}[]
\NormalTok{cust }\OtherTok{\textless{}{-}} \FunctionTok{rexp}\NormalTok{(trials,}\DecValTok{5}\NormalTok{)}
\FunctionTok{mean}\NormalTok{(cust)}
\end{Highlighting}
\end{Shaded}

\begin{verbatim}
## [1] 0.1999709
\end{verbatim}

\begin{Shaded}
\begin{Highlighting}[]
\FunctionTok{mean}\NormalTok{(cust}\SpecialCharTok{\textless{}}\NormalTok{(}\DecValTok{10}\SpecialCharTok{/}\DecValTok{60}\NormalTok{))}
\end{Highlighting}
\end{Shaded}

\begin{verbatim}
## [1] 0.56627
\end{verbatim}

\begin{enumerate}
\def\labelenumi{\alph{enumi}.}
\setcounter{enumi}{4}
\tightlist
\item
  Geometric
\end{enumerate}

\begin{Shaded}
\begin{Highlighting}[]
\NormalTok{coin }\OtherTok{\textless{}{-}} \FunctionTok{rgeom}\NormalTok{(trials,.}\DecValTok{5}\NormalTok{)}
\FunctionTok{mean}\NormalTok{(coin)}
\end{Highlighting}
\end{Shaded}

\begin{verbatim}
## [1] 0.99737
\end{verbatim}

\begin{Shaded}
\begin{Highlighting}[]
\FunctionTok{mean}\NormalTok{(coin}\SpecialCharTok{\textless{}}\DecValTok{4}\NormalTok{)}
\end{Highlighting}
\end{Shaded}

\begin{verbatim}
## [1] 0.938
\end{verbatim}

\begin{enumerate}
\def\labelenumi{\alph{enumi}.}
\setcounter{enumi}{5}
\tightlist
\item
  Normal, mean of 98.6F, var of \(100F - 98.6F = 1.4F\)
\end{enumerate}

\end{document}
